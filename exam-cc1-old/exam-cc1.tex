% Written on Tue 26 Mar 2019 15:01:06 CET
% by Jean-Baptiste Caillau, Universite Cote d'Azur, CNRS, Inria, LJAD
\documentclass[11pt,a4paper]{article}
\usepackage{hyperref}
\usepackage{amsmath}
\usepackage{mathrsfs}
\usepackage[french]{babel}
\usepackage{wasysym}
\usepackage{graphicx}
\usepackage{tp}
\usepackage{version}
\def\N{\mathbf{N}}
\def\Z{\mathbf{Z}}
\def\Q{\mathbf{Q}}
\def\R{\mathbf{R}}
\def\C{\mathbf{C}}
\def\CC{\mathscr{C}}
\def\EE{\mathscr{E}}
\def\T{\mathbf{T}}
\def\K{\mathbf{K}}
\def\L{\mathrm{L}}
\def\H{\mathrm{H}}
\def\W{\mathrm{W}}
\def\M{\mathrm{M}}
\def\O{\mathrm{O}}
\def\Im{\mathrm{Im}}
\def\Vect{\mathrm{Vect}}
\def\Min{\mathrm{Min}}
\def\BV{\mathrm{BV}}
\def\Isom{\mathrm{Isom}}
\def\iy{\infty}
\def\d{\mathrm{d}}
\def\t{\ \!^t\!}
\def\tr{\mathrm{tr}}
\def\veps{\varepsilon}
\def\vphi{\varphi}
\def\la{\langle}
\def\ra{\rangle}
\def\noi{\noindent}
\def\cf{\emph{cf.}}
\def\ie{\emph{i.e.}}
\def\etc{\emph{etc.}}
\renewcommand{\tilde}{\widetilde}
\renewcommand{\hat}{\widehat}
\theoremstyle{plain}
\newtheorem{thrm}{Th\'eor\`eme}[section]
\newtheorem{prpstn}{Proposition}[section]
\newtheorem{lmm}{Lemme}[section]
\newtheorem{crllr}{Corollaire}[section]
\newtheorem{dfntn}{D\'efinition}[section]
\theoremstyle{definition}
\newtheorem{rmrk}{Remarque}[section]

%\excludeversion{corr}
\includeversion{corr}

\title{Exam CC no.~1}
\shorttitle{Examen CC no.~1}
\numero{Examen CC no.~1}
\date{2018--2019}
\discipline{MI2}
\promotion{Polytech Nice Sophia --- MAM3}

\begin{document}
\maketitle

{\bf Dur\'ee 1H30. Tous les exercices sont ind\'ependants.
Le bar\`eme pr\'e\-vi\-sion\-nel est indiqu\'e pour chaque exercice.
%Rendre sur des copies s\'epar\'ees l'exercice 1 d'une part, les exercices 2 et 3
%d'autre part.
Documents autoris\'es~: une feuille de notes de cours recto-verso manuscrite.}

% Exo 1
\begin{Exercice}[3 points]
Montrer que l'application $f : \R^2 \to \R^3$ d\'efinie par
\[ f(x_1,x_2) := \left[ \begin{array}{c}
  x_1x_2\exp(x_1)-x_2^2\\
  x_2^2\cos(x_1-x_2^2)\\
  x_1+x_2
 \end{array} \right] \]
est d\'erivable et donner l'expression de sa d\'eriv\'ee.\\

\begin{corr} $\RHD$ Les d\'eriv\'ees partielles de chaque composante de l'application
existent et sont continues, la fonction est donc de classe $\CC^1$ (et en particulier
d\'erivable). On a
\[ f'(x) =
  \left[ \begin{array}{ccc}
  (x_2+x_1x_2)\exp(x_1) & x_1\exp(x_1)-2x_2\\  
  -x_2^2\sin(x_1-x_2^2) & 2x_2\cos(x_1-x_2^2)+2x_2^3\sin(x_1-x_2^2)\\ 
  1 & 1
  \end{array} \right]. \]
\end{corr}
\end{Exercice}

% Exo 2
\begin{Exercice}[4 points]
\begin{Question}
Montrer que l'application $f : \R^n \to \R$ d\'efinie par
\[ f(x) := \ln(1+\|x\|^2) \]
est d\'erivable et donner l'expression de son gradient. La norme d\'esigne la norme
euclidienne sur $\R^n$,
\[ \|x\|=(|x_1|^2+\cdots+|x_n|^2)^{1/2}. \]
\end{Question}

\begin{corr} $\RHD$ L'application est d\'erivable comme compos\'ee d'applications
d\'erivables, $x \mapsto \|x\|^2 \mapsto \ln(1+\|x\|^2)$, d'o\`u
\[ \nabla f(x) = \frac{2x}{1+\|x\|^2}\cdot \]
\end{corr}

\begin{Question} Montrer que $\nabla f$ est \'egalement d\'erivable et donner
l'expression du hessien de $f$.
\end{Question}

\begin{corr} $\RHD$
Le gradient est d\'erivable comme produit (vecteur $\times$ scalaire,
bilin\'eaire) et composition d'applications d\'erivables, et
\[ \nabla^2 f(x) = \frac{2(1+\|x\|^2)I-4x\,^t x}{(1+\|x\|^2)^2} \cdot \]
\end{corr}
\end{Exercice}

% Exo 3
\begin{Exercice}[6 points]
\begin{Question} 
Mettre l'\'equation diff\'erentielle
\[ \dot{x}(t)=x(t)-1 \]
sous la forme $\dot{x}(t)=f(t,x(t))$ avec $f:\R^2 \to \R$ que l'on pr\'ecisera.
\end{Question}

\begin{corr} $\RHD$ 
\[ f(t,x) = x-1 \]
\end{corr}

\begin{Question}
\'Etant donn\'e $(t_0,x_0) \in \R^2$,
justifier que l'\'equation diff\'erentielle
\[ \dot{x}(t)=x(t)-1,\quad x(t_0)=x_0, \]
poss\`ede une unique solution maximale. 
\end{Question}

\begin{corr} $\RHD$ L'application $f$ est 
de classe $\CC^1$, donc le th\'eor\`eme des accroissements finis permet d'affirmer
qu'elle est localement lipschitzienne en $x$. Le th\'eor\`eme de Cauchy-Lipschitz
s'applique donc et garantit l'existence et l'unicit\'e de solution maximale pour toute
condition initiale.
\end{corr}

\begin{Question} D\'eterminer la solution maximale de l'\'equation diff\'erentielle
\[ \dot{x}(t)=x(t)-1,\quad x(0)=1. \]
\end{Question}

\begin{corr} $\RHD$ $x(t)=1$, pour tout $t \in \R$
\end{corr}

\begin{Question} On consid\`ere l'\'equation diff\'erentielle
\[ \dot{x}(t)=x(t)-1,\quad x(0)=2. \]
Justifier que sa solution maximale est toujours strictement sup\'erieure \`a $1$,
et d\'eterminer cette solution maximale.
\end{Question}

\begin{corr} $\RHD$ Si la solution prenait la valeur $1$, elle serait identiquement
\'egale \`a $1$ (ce qui est interdit par la condition intiale). Donc $x(t) \neq 1$
pour tout $t$ de l'intervalle de d\'efinition
(en vertu de l'unicit\'e de solution) et, par continuit\'e,
soit $x(t)>1$, soit $x(t)<1$. Vu la condition initiale, on a constamment $x(t)>1$.
On peut donc
r\'esoudre le probl\`eme pos\'e en divisant pour "s\'eparer les variables"~:
\[ \d x/(x-1) = \d t, \]
soit, en int\'egrant,
\[ \ln\left| \frac{x(t)-1}{x(0)-1} \right| = \ln(x(t)-1) = t, \]
%ce qui implique en particulier $t \neq 1$. On obtient
On obtient
\[ x(t) = e^t+1,\quad t \in \R, \]
qui est n\'ecessairement maximale
puisque d\'efinie sur tout $\R$.
%puisque l'intervalle de d\'efinition doit contenir $t=0$ sans contenir $t=1$.
\end{corr}
\end{Exercice}

% Exo 4
\begin{Exercice}[7 points]
On rappelle que $\ell^p$, l'ensemble des suites r\'eelles de puissance $p$-i\`eme
sommable, est d\'efini comme suit~:
\[ \ell^p := \{(x_k)_k \in \R^\N\ |\ \sum_{k=0}^\iy |x_k|^p < \iy \},\quad
   p \geq 1. \]

\begin{Question}
Montrer qu'on d\'efinit une norme sur $\ell^1$ en posant
\[ \|(x_k)_k\|_1 = \sum_{k=0}^\iy |x_k|. \]
\end{Question}

\begin{corr} $\RHD$ (i) La positivit\'e est \'evidente, et $\sum_k |x_k|=0$ implique
$|x_k|=0$ pour tout $k$ (s\'erie \`a termes positifs), donc $(x_k)_k$ est bien la
suite nulle~; (ii) soit $\lambda \in \R$, $\sum_k |\lambda x_k|=\lim_{K \to \iy}
|\lambda| \sum_{k=0}^K |x_k| = |\lambda| \sum_k |x_k|$~; (iii) $\sum_k |x_k+y_k| \leq
\sum_k (|x_k|+|y_k|) = \sum_k |x_k| + \sum_k |y_k|$.
\end{corr}

\begin{Question} La suite $(x_k)_k$ de terme g\'en\'eral $x_k=1/(k+1)$, $k \in \N$,
appartient-elle \`a $\ell^1$~? \`A $\ell^2$~? Qu'en d\'eduit-on~?
\end{Question}

\begin{corr} $\RHD$
D'apr\`es le crit\`ere de Riemann, la suite est dans $\ell^2$, pas dans $\ell^1$. On
sait d'apr\`es le TD que $\ell^1 \subset \ell^2$, l'inclusion est donc stricte.
\end{corr}

%% \begin{Question} Soit $(X_n)_n \in (\ell^1)^\N$ une suite de suites de $\ell^1$. Pour
%% $n \in N$, on note $x_{n,k}$ le terme g\'en\'eral de la suite $X_n$~:
%% $X_n=(x_{n,k})_k \in \ell^1$, $n \in N$. Donner la d\'efinition de~: "La suite
%% $(X_n)_n$ est de Cauchy dans $\ell^1$".
%% \end{Question}
%% 
%% \begin{corr} $\RHD$ En passant \`a la limite sur $q$, on obtient (par continuit\'e de
%% la valeur absolue)
%% \[ (\forall \veps>0)(\exists N \in \N)(\forall p \geq N)(\forall x \in A):
%%    |f_p(x)-f(x)| \leq \veps. \]
%% Pour $\veps=1$, il existe $N_1 \in \N$ tel que
%% \[ (\forall x \in A): |f_p(x)-f(x)| \leq 1, \]
%% ce qui montre que $f-f_p$ est born\'ee, et donc que $f=f_p+(f-f_p)$ appartient
%% \'egalement \`a $E$ (espace vectoriel).
%% \end{corr}

\begin{Question} On d\'efinit la suite $(X_n)_n \in (\ell^1)^\N$ en posant
\[ X_n := (\underbrace{1,\dots,1}_{n \text{ fois}},0,0,0,\dots), \]
c'est \`a dire en posant
\begin{align*}
  X_0 &:= (0,0,0,0,0,\dots),\\
  X_1 &:= (1,0,0,0,0,\dots),\\
  X_2 &:= (1,1,0,0,0,\dots),\\
  X_3 &:= (1,1,1,0,0,\dots),\\
  \vdots
\end{align*}
Justifier que, pour tout $n \in \N$, $X_n$ appartient effectivement \`a $\ell^1$.
\end{Question}

\begin{corr} $\RHD$ Chacune des suites est presque nulle (seul un nombre fini de
termes sont non nuls), donc chacune d'elles appartient \`a $\ell^1$. 
\end{corr}

\begin{Question} Pour $n \in \N$, calculer $\|X_{n+1}-X_n\|_1$.
La suite $(X_n)_n$ est-elle de Cauchy dans $\ell^1$~? 
La suite $(X_n)_n$ converge-t-elle dans $\ell^1$~? 
\end{Question}

\begin{corr} $\RHD$ On a $\|X_{n+1}-X_n\|_1=1$, la suite n'est donc pas de Cauchy. Elle
n'est donc pas convergente.
\end{corr}

\end{Exercice}

%\vfill \begin{flushright}{\footnotesize \emph{En ligne sous}
%\texttt{caillau.perso.math.cnrs.fr/mi2}} \end{flushright}
\end{document}

\begin{corr} $\RHD$
\end{corr}
