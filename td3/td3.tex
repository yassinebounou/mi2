% Written on Wed 21 Mar 2018 09:39:26 CET
% by Jean-Baptiste Caillau, LJAD, Univ. Cote d'Azur & CNRS/Inria
\documentclass[11pt,a4paper]{article}
\usepackage{hyperref}
\usepackage{amsmath}
\usepackage{mathrsfs}
\usepackage{graphicx}
\usepackage[latin1]{inputenc}
\usepackage{tp}
\def\N{\mathbf{N}}
\def\Z{\mathbf{Z}}
\def\Q{\mathbf{Q}}
\def\R{\mathbf{R}}
\def\C{\mathbf{C}}
\def\K{\mathbf{K}}
\def\L{\mathrm{L}}
\def\H{\mathrm{H}}
\def\W{\mathrm{W}}
\def\iy{\infty}
\def\d{\mathrm{d}}
\def\t{\ \!^t\!}
\def\veps{\varepsilon}
\def\vphi{\varphi}
\def\la{\langle}
\def\ra{\rangle}
\def\noi{\noindent}
\def\cf{\emph{cf.}}
\def\ie{\emph{i.e.}}
\renewcommand{\tilde}{\widetilde}
\renewcommand{\hat}{\widehat}

\title{TD~3 -- Espaces de suites}
\shorttitle{TD~3}
\numero{TD~3}
\date{2019--2020}
\discipline{MI2}
\promotion{Polytech Nice Sophia --- MAM3}

\begin{document}
\maketitle

Pour $1 \leq p < \iy$ on d\'efinit $\ell^p$ l'ensemble des suites r\'eelles
de puissance $p$-i\`eme sommable, et $\ell^\iy$ l'ensemble des suites r\'eelles born\'ees.
(Les d\'efinitions et r\'esultats qui suivent valent aussi pour des suites complexes.)

% Exercice 1
\begin{Exercice}
Montrer qu'on
d\'efinit une norme sur $\ell^p$ en posant
\[ \|(x_k)_k\|_p = (\sum_k |x_k|^p)^{1/p},\quad 1 \leq p < \iy, \]
et
\[ \|(x_k)_k\|_\iy = \sup_k |x_k|. \]
\end{Exercice}

% Exercice 2
\begin{Exercice} Soient $1 \leq p < q \leq \infty$.
Montrer que $\ell^p \subsetneq \ell^q$, avec
injection continue.
\end{Exercice} \vspace*{1em}

% Exercice 3
\begin{Exercice} Montrer que $(\ell^p,\|.\|_p)$ est un Banach, $1 \leq p
\leq \iy$. \end{Exercice} \vspace*{1em}

%% Exercice 3
%\begin{Exercice} Soit $1<p<\iy$ et soit $p'$ son exposant conjugu\'e
%($\frac{1}{p}+\frac{1}{p'}=1$).
%\begin{Question} Soit $Y=(y_k)_k \in \ell^{p'}$, montrer que
%\begin{equation} \label{eq1}
%\vphi_Y\ :\ \begin{array}{rcl}
%  \ell^p & \rightarrow & \K\\
%  X=(x_k)_k & \mapsto & \sum_k x_k y_k
%\end{array}
%\end{equation}
%est une forme lin\'eaire continue. \end{Question}
%
%\begin{Question} R\'eciproquement, soit $\vphi \in (\ell^p)'$, montrer
%qu'il existe $Y \in \ell^{p'}$ tel que $\vphi =
%\vphi_Y$ (telle que d\'efinie en (\ref{eq1})). \end{Question}
%
%\begin{Question} Conclure en montrant
%que $(\ell^p)'$ et $\ell^{p'}$ sont isomorphes et isom\'etriques. \end{Question}
%\end{Exercice} \vspace*{1em}

\vfill \begin{flushright}{\footnotesize \emph{En ligne sous}
\texttt{caillau.perso.math.cnrs.fr/mi2}} \end{flushright}

\end{document}
