% Written on Mon 28 May 2018 06:44:40 CEST
% by Jean-Baptiste Caillau, LJAD, Univ. Cote d'Azur & CNRS/Inria
\documentclass[11pt,a4paper]{article}
\usepackage{hyperref}
\usepackage{amsmath}
\usepackage{mathrsfs}
\usepackage{graphicx}
\usepackage[latin1]{inputenc}
\usepackage{tp}
\def\N{\mathbf{N}}
\def\Z{\mathbf{Z}}
\def\Q{\mathbf{Q}}
\def\R{\mathbf{R}}
\def\C{\mathbf{C}}
\def\K{\mathbf{K}}
\def\L{\mathrm{L}}
\def\H{\mathrm{H}}
\def\M{\mathrm{M}}
\def\W{\mathrm{W}}
\def\O{\mathrm{O}}
\def\tr{\mathrm{tr}}
\def\ch{\mathrm{ch}}
\def\sh{\mathrm{sh}}
\def\CC{\mathscr{C}}
\def\DD{\mathscr{D}}
\def\vp{\mathrm{vp}}
\def\Vect{\mathrm{Vect}}
\def\iy{\infty}
\def\d{\mathrm{d}}
\def\t{\ \!^t\!}
\def\veps{\varepsilon}
\def\vphi{\varphi}
\def\la{\langle}
\def\ra{\rangle}
\def\noi{\noindent}
\def\cf{\emph{cf.}}
\def\ie{\emph{i.e.}}
\renewcommand{\tilde}{\widetilde}
\renewcommand{\hat}{\widehat}

\title{TD~7 -- Probl\`emes aux limites}
\shorttitle{TD~7}
\numero{TD~7}
\date{2019--2020}
\discipline{MI2}
\promotion{Polytech Nice Sophia --- MAM3}

\begin{document}
\maketitle

% Exercice 1
\begin{Exercice} On consid\`ere le sous-espace vectoriel $\H^1(]0,1[)$ de $\L^2(]0,1[)$
des classes de fonctions dont la d\'eriv\'ee au sens des distributions appartient
encore \`a $\L^2(]0,1[)$~: $u \in \H^1(]0,1[)$ si et seulement s'il existe $v \in
\L^2(]0,1[)$ telle que, pour tout $\vphi \in \DD(]0,1[)$,
\[ \int_0^1 u\vphi'\,\d t =-\int_0^1 v\vphi\,\d t. \]

\begin{Question} Montrer qu'on d\'efinit un produit scalaire sur $\H^1(]0,1[)$ en
posant
\[ (u|v)_{H^1} := (u|v)_{\L^2}+(u'|v')_{\L^2}. \]
\end{Question}

\begin{Question} Montrer que $\H^1(]0,1[)$, muni de ce produit scalaire, est un espace
de Hilbert.
\end{Question}

\begin{Question} En utilisant le fait que pour tout $u$ dans $\H^1(]0,1[)$
on a (existence d'un repr\'esentant continu tel que)
\[ u(t) = u(0) + \int_0^t u'(t)\,\d t,\quad t \in [0,1], \]
montrer que le sous-espace vectoriel $\H^1_0(]0,1[)$ de $\H^1(]0,1[)$ des (classes de)
fonctions $u$ telles que $u(0)=u(1)=0$ est \'egalement un espace de Hilbert pour ce
produit scalaire.
\end{Question}

\end{Exercice} \vspace*{1em}

% Exercice 2
\begin{Exercice} On consid\`ere le probl\`eme avec conditions aux limites \emph{de
Dirichlet} suivant~: trouver $u \in \CC^2([0,1])$ telle que
\[ -u''(t)+u(t) = f(t),\quad t \in ]0,1[, \]
\[ u(0)=0,\quad u(1)=0, \]
o\`u $f$ est une fonction donn\'ee de $\CC^0([0,1])$.

\begin{Question} Montrer que toute solution ("forte") $u$ de ce probl\`eme
est \'egalement solution ("faible") de l'\'equation suivante~: quel que soit $v \in
\H^1_0(]0,1[)$,
\[ (u|v)_{H^1} = \int_0^1 fv\,\d t. \]
\end{Question}

\begin{Question} Montrer qu'on a existence et unicit\'e de solution faible dans
$\H^1_0(]0,1[)$.
\end{Question}

\begin{Question} Montrer que, si $f \in \CC^0([0,1])$, la solution faible appartient
\`a $\CC^2([0,1])$.
\end{Question}

\begin{Question} En d\'eduire que, si $f \in \CC^0([0,1])$, toute solution faible est
aussi solution forte.
\end{Question}

\end{Exercice} \vspace*{1em}

% Exercice 3
\begin{Exercice} On consid\`ere le probl\`eme avec conditions aux limites
\emph{mixtes} suivant~: trouver $u \in \CC^2([0,1])$ telle que
\[ -u''(t)+u(t) = f(t),\quad t \in ]0,1[, \]
\[ u(0)=0,\quad u'(1)=0, \]
o\`u $f$ est une fonction donn\'ee de $\CC^0([0,1])$. Proposer une formulation
varia\-tionnelle de ce probl\`eme, puis r\'esoudre.

\end{Exercice} \vspace*{1em}

\vfill \begin{flushright}{\footnotesize \emph{En ligne sous}
\texttt{caillau.perso.math.cnrs.fr/mi2}} \end{flushright}

\end{document}

\begin{Exercice}

\begin{Question} 
\end{Question}
