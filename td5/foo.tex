## Exercice 1
Fonction $\zeta$ de Riemann

Soit
$$ \zeta(s) = \sum_{n \geq 1} \frac{1}{n^s},\ s \in ]1,+\iy[. $$
Calculer $\zeta(2)$, $\zeta(4)$ et $\zeta(6)$. 

## Exercice 2
Soit $f \in \mathbf{R}^\mathbf{R}$, paire et $2\pi$-périodique définie par
$f(x)=\pi-2x$ sur $[0,\pi[$.

### 2.1
Donner l'expression de la série de Fourier de $f$
sur la base hilbertienne des polynômes trigonométriques.

### 2.2
Indiquer la nature de la convergence de la série de
Fourier de $f$.

### 2.3
En déduire
$$ \sum_{p \geq 0} \frac{1}{(2p+1)^2}\ \text{et}\ 
   \sum_{p \geq 0} \frac{1}{(2p+1)^4}\cdot$$

## Exercice 3
Soit $f \in \mathbf{R}^\mathbf{R}$ $2\pi$-périodique définie par
$f(x)=e^{ax}$ sur $[0,2\pi[$ ($a \neq 0$).

### 3.1
Donner l'expression de la série de Fourier de $f$
sur la base hilbertienne des polynômes trigonométriques.

### 3.2
Indiquer la nature de la convergence de la série de
Fourier de $f$.

### 3.3
En déduire
$$ \sum_{n \geq 1} \frac{a}{a^2+n^2}\cos nx\ \text{et}\ 
   \sum_{n \geq 1} \frac{n}{a^2+n^2}\sin nx. $$

### 3.4
En déduire également
$$ \sum_{n \geq 1} \frac{1}{(a^2+n^2)^2}\ \text{et}\ 
   \sum_{n \geq 1} \frac{n^2}{(a^2+n^2)^2}. $$
(Rappel : $\ch\,x=(e^x+e^{-x})/2$ et $\sh\,x=(e^x-e^{-x})/2$.)

## Exercice 4
Soit $E$ l'ensemble des (classes de) fonctions
mesurables $x:[0,1] \to \mathbf{R}$ telles que
$$ \int_{[0,1]} \frac{|x(t)|^2}{t}dt < \iy. $$

### 4.1
Montrer que $E$ est un espace vectoriel
contenant les fonctions nulles et dérivables à l'origine.

### 4.2
Montrer que
$$ (x|y) = \int_{[0,1]} \frac{x(t)y(t)}{t}dt $$
définit un produit scalaire sur $E$. \end{Question}

### 4.3
On note $(P_n)_{n \geq 1}$ le SON obtenu par
orthonormalisation de $\mathbf{R}[X] \backslash \mathbf{R}$. Calculer $P_i$,
$i=1,\dots,3$.
